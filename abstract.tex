\documentclass[a4paper, 12pt]{article}

\begin{document}

\section{Securecodeing}
\subsection{Stide}
\subsection{Security Adjustation}
\subsection{Security Assurence Level}

\section{WebSec}
\subsection{Black and Whitebox testing}
	\large Blackbox:
	Testumgebung bei der wenig bis keine Informationen über das System vorliegen.
	\large Whitebox:
	Testumgebung bei der genauere Informationen über das zu testende System vorliegen.
\subsection{OWASP Top 10}
	\large OWASP (Open Web Application Security Project)
	\newbox
	\large Goals:
	\begin{itemize}
		\item "identifying some of the most critical risks"
		\item "to raise awareness about application security"
		\item "not an application security program"
	\end{itemize}
	\newbox
	\large Top 10:
	\begin{enumerate}
		\item Injection
		\item XSS
		\item Broken Authentication and Session Management
		\item Insecure ddirect object references
		\item XS request forgery
		\item Security misconfiguration
		\item Insecure cryptographic storage
		\item Failure to restrict URL access
		\item Insufficient transport layer protection
		\item Invalidated redirects and forwards
	\end{enumerate}
\subsection{SQL Injection}
One of the most exploited vuln., despite being well known for almost a decade
\begin{itemize}
\item Problem: App doesn't properly separate database statements form user data
\item Impact: Read/Write access to database (sometimes even databases form other apps which are on the same server)
\item Tool support: sqlmap (can find injection, dump content, supports different DBMS)
\item Solution: Never trust user-supplied data - Separate query logic form user data
\item Good Practice: Prepared statements
\end{itemize}
\subsection{REST}

\section{WebSec 2}
\subsection{SQL Injection}
\subsection{XSS}

\section{Security Challanges}
\subsection{Attackermodel}
\subsection{Standartmodel}
\subsection{Dolyveyao}
\subsection{NithemSchroeder PublicKey}
SYNATX
\subsection{Attacks Generel}
zb Social Eng ...
\subsection{MathShit ...}
\subsection{Muenzwurf}
\subsection{Dataflow}
\subsection{Polices}
\subsection{Examples for Non-Modularity}

\section{SmartGrid}
\subsection{Szenario}
\subsection{Which Attacks are Posible}
---Only highlevel
\subsection{Abrechnung und Steuerung im Privathaus}

\section{Internet of Things}
\subsection{Applikationbeispiele}
\subsection{Sec Challanges}
//TODO combine the following points better
\begin{itemize}
\item larger attack surface
\begin{itemize}
\item large number of interfaces
\item new possibilities of exploiting existing vuln. and threats
\item new interaction possibilities
\item new possible attack patterns or procedures, threats and damages
\item IoT devices use local and remote services and provide services with or without any human intervertion
\end{itemize}
\item Fault Tolerance
\begin{itemize}
\item Connectivity is not constant
\item context of devices changes over timem sometimes abruptly
\item system must cope with these changes, providing a partial service until the conditions are more facorable
\end{itemize}
\item Efficient Cryptographic Primitces
\begin{itemize}
\item provide good secirity primitives suitable for low resource consumption(energy, time, sace)
\item deployment of efficient key distribution and management systems
\end{itemize}
\item Integrate IoT with the rest of the Internet
\begin{itemize}
\item use protocols and security mechanisms used in the internet or bridge to them, archieving end-to-end security when translating the different underlying protocols, existence of such sub-networks and interactions between them must be taken into account
\end{itemize}
\item Intrusion Detection Systems and Survivability Mechanisms
\begin{itemize}
\item react to changes in environment
\end{itemize}
\item Trust Management and Secure Collaboration
\begin{itemize}
\item Need trust relationships between users and devices
\item Devices might know each other, or might be complete stranges
\item Need to interoperate the security policies of different components(how will devices react if a particular event arises?, Who are the elements i should collaborate with, and how?)
\end{itemize}
\item Models and mechanisms for device and data ownership
\begin{itemize}
\item Users may utilize their devices to authenticate themselves
\item Devices might perform certain operations on behalf of their users
\item MEchanisms and security policies and mechanisms that control how the data is created, accessed and protected
\end{itemize}
\item Software maintenance process
\begin{itemize}
\item How to release security patches
\end{itemize}
\item Usability
\begin{itemize}
\item security for IoT-based applications must be understandable and manageable by end-users
\end{itemize}
\item Privacy
\begin{itemize}
\item "Things" belong to people and collect information about actions of them
\item Devices and interfaces should do not leak personal information aboit the users location activities preferences // haha nice try NSA :D
\item Need to avoid that a communication partner is able to collect large amounts of information. Current Algorithms do not prevent this type of attack
\end{itemize}
\end{itemize}

\subsection{Sec Solutions}
\subsection{RFID}
\subsection{Blockerback}
\subsection{Baumbasierendes Model}
fuer RFID
\section{6 Threads Landscape}
\subsection{Phishing}
\subsection{Spearphising}
\subsection{Zertificate}
\subsection{Chain of Trust}

\section{Cause Analysis}
\subsection{Implementation Errors}
die ueber legitime Ports ausgenutzt werden koennen.

\section{Product ...}
\subsection{Common criteria}
\subsection{Was ist ein Sec Target}
\subsection{Was ist ein Connection Profile}
\subsection{Security function requ}
\subsection{Security assurance requ}
\subsection{Was sind EALs}


\section{WebVulnerabilities}
\subsection{State of the art}
\subsection{Top 10}
Already in section WebSec
\subsection{SQL Injection Counter MEasures}
SYNTAX
Some Stuff in section websec 
//TODO add syntax
//TODO add example Counter Measures
\subsection{XSS - Cross-site scripting}
\begin{itemize}
\item User injects JavaScript code that subsequentially gets executed in the browser of the visitor
\item Problem: User data that contains HTML markup is not properly escaped, the browser renders the attacker code as port of the web page
\end{itemize}

\subsection{CSRF}
\subsection{Session Hacking}
//Broken Authentication and session mgmt.
\begin{itemize}
\item Badly implemented access control: index.html redirects to secret.html if the password is correct
\item Sometimes one can go directly to secret.html without password check
\end{itemize}

\section{Formal Security Models}
\subsection{Welche art von Modellen gibt es}
\subsection{Ansatz und Vorteile von Formal Security}
\subsection{Conclusions}

\section{BufferOverflow}
\subsection{Buffer Overflows}
Things good to know:
\begin{itemize}
\item First publication 1972
\item First documented exploit, Morris Worm 1988
\item Three famous Worms(2001 Code Red Worm, 2003 SQL Slammer, 2008 Conficker Worm)
\item Triggered by external data input with Dangerous code
\item Mostly seen in C/C++ because of missing memory boundarys or access checking
\end{itemize}
\subsubsection{Stack Buffer Overflows}
Programm Stack: used for managing prog execution and prog state, saves 
\begin{itemize}
\item Return Address
\item Function Arguments
\item Local Varibales
\end{itemize}
on the Stack Frame.
Register EBP saves actual address of the Frame (Intel)
Frame Pointer is the Reference Pointer within the Stack
Stack gets modified during following actions
\begin{itemize}
\item Function call
\item Function Initialization
\item When function returns
\end{itemize}
//TODO merge FunctionCall example
\subsubsection{Heap Buffer Overflows}
Ich denke hierzu haben wir nicht wirklich was gemacht weis nicht ob das jedoch bekannt ist 
\subsection{Integer Overflows}
\subsection{Format String Attacks}
\subsection{Race Conditions}

\section{NOT RELEVANT}
\begin{tabular}{|c|c|} \hline
\textbf{Section} & \textbf{Thema} \\ \hline
Internet of Things & Geo Privacy \\ \hline
5 & Best Practice \\ \hline
5 & Risk for Tests \\ \hline
WebVul & extra Slides am ende \\ \hline
Smart Metering & * \\ \hline
Formal Sec Mod & Siemens zeug \\ \hline



\end{tabular}
\end{document}
