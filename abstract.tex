\documentclass[a4paper, 12pt]{article}

\begin{document}
\section{Securecodeing}
\subsection{Stide}
\subsection{Security Adjustation}
\subsection{Security Assurence Level}

\section{WebSec}
\subsection{Black and Whitebox testing}
	\large Blackbox:
	Testumgebung bei der wenig bis keine Informationen über das System vorliegen.
	\large Whitebox:
	Testumgebung bei der genauere Informationen über das zu testende System vorliegen.
\subsection{OWASP Top 10}
	\large OWASP (Open Web Application Security Project)
	\newbox
	\large Goals:
	\begin{itemize}
		\item "identifying some of the most critical risks"
		\item "to raise awareness about application security"
		\item "not an application security program"
	\end{itemize}
	\newbox
	\large Top 10:
	\begin{enumerate}
		\item Injection
		\item XSS
		\item Broken Authentication and Session Management
		\item Insecure ddirect object references
		\item XS request forgery
		\item Security misconfiguration
		\item Insecure cryptographic storage
		\item Failure to restrict URL access
		\item Insufficient transport layer protection
		\item Invalidated redirects and forwards
	\end{enumerate}
\subsection{SQL Injection}
\subsection{REST}

\section{WebSec 2}
\subsection{SQL Injection}
\subsection{XSS}

\section{Security Challanges}
\subsection{Attackermodel}
\subsection{Standartmodel}
\subsection{Dolyveyao}
\subsection{NithemSchroeder PublicKey}
SYNATX
\subsection{Attacks Generel}
zb Social Eng ...
\subsection{MathShit ...}
\subsection{Muenzwurf}
\subsection{Dataflow}
\subsection{Polices}
\subsection{Examples for Non-Modularity}

\section{SmartGrid}
\subsection{Szenario}
\subsection{Which Attacks are Posible}
---Only highlevel
\subsection{Abrechnung und Steuerung im Privathaus}

\section{Internet of Things}
\subsection{Applikationbeispiele}
\subsection{Sec Challanges}
\subsection{Sec Solutions}
\subsection{RFID}
\subsection{Blockerback}
\subsection{Baumbasierendes Model}
fuer RFID
\section{6 Threads Landscape}
\subsection{Phishing}
\subsection{Spearphising}
\subsection{Zertificate}
\subsection{Chain of Trust}

\section{Cause Analysis}
\subsection{Implementation Errors}
die ueber legitime Ports ausgenutzt werden koennen.

\section{Product ...}
\subsection{Common criteria}
\subsection{Was ist ein Sec Target}
\subsection{Was ist ein Connection Profile}
\subsection{Security function requ}
\subsection{Security assurance requ}
\subsection{Was sind EALs}


\section{WebVulnerabilities}
\subsection{State of the art}
\subsection{Top 10}
\subsection{SQL Injection Counter MEasures}
SYNTAX
\subsection{XSS}
\subsection{CSRF}
\subsection{Session Hacking}

\section{Formal Security Models}
\subsection{Welche art von Modellen gibt es}
\subsection{Ansatz und Vorteile von Formal Security}
\subsection{Conclusions}

\section{BufferOverflow}

\section{NOT RELEVANT}
\begin{tabular}{|c|c|} \hline
\textbf{Section} & \textbf{Thema} \\ \hline
Internet of Things & Geo Privacy \\ \hline
5 & Best Practice \\ \hline
5 & Risk for Tests \\ \hline
WebVul & extra Slides am ende \\ \hline
Smart Metering & * \\ \hline
Formal Sec Mod & Siemens zeug \\ \hline



\end{tabular}
\end{document}
