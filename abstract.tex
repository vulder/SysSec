\documentclass[a4paper, 12pt]{article}

\begin{document}

\section{Securecodeing}
\subsection{Stide}
\subsection{Security Adjustation}
\subsection{Security Assurence Level}

\section{WebSec}
\subsection{Black and Whitebox testing}
\subsection{OWASP Top 10}
\begin{itemize}
\item A01 Injection
\item A02 Cross-site scripting
\item A03 Broken Authentication and Session Mgmt.
\item A04 Insecure direct object references
\item A05 Cross-site request forgery
\item A06 Security misconfiguration
\item A07 Insecure cryptographic storage
\item A08 Failure to restrict URL access
\item A09 Insufficient transport layer protection
\item A10 Invalidated redirects and forwards
\end{itemize}
\subsection{SQL Injection}
\subsection{REST}

\section{WebSec 2}
\subsection{SQL Injection}
\subsection{XSS}

\section{Security Challanges}
\subsection{Attackermodel}
\subsection{Standartmodel}
\subsection{Dolyveyao}
\subsection{NithemSchroeder PublicKey}
SYNATX
\subsection{Attacks Generel}
zb Social Eng ...
\subsection{MathShit ...}
\subsection{Muenzwurf}
\subsection{Dataflow}
\subsection{Polices}
\subsection{Examples for Non-Modularity}

\section{SmartGrid}
\subsection{Szenario}
\subsection{Which Attacks are Posible}
---Only highlevel
\subsection{Abrechnung und Steuerung im Privathaus}

\section{Internet of Things}
\subsection{Applikationbeispiele}
\subsection{Sec Challanges}
\subsection{Sec Solutions}
\subsection{RFID}
\subsection{Blockerback}
\subsection{Baumbasierendes Model}
fuer RFID
\section{6 Threads Landscape}
\subsection{Phishing}
\subsection{Spearphising}
\subsection{Zertificate}
\subsection{Chain of Trust}

\section{Cause Analysis}
\subsection{Implementation Errors}
die ueber legitime Ports ausgenutzt werden koennen.

\section{Product ...}
\subsection{Common criteria}
\subsection{Was ist ein Sec Target}
\subsection{Was ist ein Connection Profile}
\subsection{Security function requ}
\subsection{Security assurance requ}
\subsection{Was sind EALs}


\section{WebVulnerabilities}
\subsection{State of the art}
\subsection{Top 10}
\subsection{SQL Injection Counter MEasures}
SYNTAX
\subsection{XSS}
\subsection{CSRF}
\subsection{Session Hacking}

\section{Formal Security Models}
\subsection{Welche art von Modellen gibt es}
\subsection{Ansatz und Vorteile von Formal Security}
\subsection{Conclusions}

\section{BufferOverflow}
\subsection{Buffer Overflows}
Things good to know:
\begin{itemize}
\item First publication 1972
\item First documented exploit, Morris Worm 1988
\item Three famous Worms(2001 Code Red Worm, 2003 SQL Slammer, 2008 Conficker Worm)
\item Triggered by external data input with Dangerous code
\item Mostly seen in C/C++ because of missing memory boundarys or access checking
\end{itemize}
\subsubsection{Stack Buffer Overflows}
Programm Stack: used for managing prog execution and prog state, saves 
\begin{itemize}
\item Return Address
\item Function Arguments
\item Local Varibales
\end{itemize}
on the Stack Frame.
Register EBP saves actual address of the Frame (Intel)
Frame Pointer is the Reference Pointer within the Stack
Stack gets modified during following actions
\begin{itemize}
\item Function call
\item Function Initialization
\item When function returns
\end{itemize}
//TODO merge FunctionCall example
\subsubsection{Heap Buffer Overflows}
Ich denke hierzu haben wir nicht wirklich was gemacht weis nicht ob das jedoch bekannt ist 
\subsection{Integer Overflows}
\subsection{Format String Attacks}
\subsection{Race Conditions}

\section{NOT RELEVANT}
\begin{tabular}{|c|c|} \hline
\textbf{Section} & \textbf{Thema} \\ \hline
Internet of Things & Geo Privacy \\ \hline
5 & Best Practice \\ \hline
5 & Risk for Tests \\ \hline
WebVul & extra Slides am ende \\ \hline
Smart Metering & * \\ \hline
Formal Sec Mod & Siemens zeug \\ \hline



\end{tabular}
\end{document}
