\documentclass[a4paper, 12pt]{article}

\begin{document}

\section{Securecodeing}
\subsection{Stide}
$ //TODO $
\subsection{Security Adjustation}
$ //TODO $
\subsection{Security Assurence Level}
$ //TODO $

\section{WebSec}
\subsection{Black and Whitebox testing}
	\large Blackbox:
	Testumgebung bei der wenig bis keine Informationen über das System vorliegen.
	\large Whitebox:
	Testumgebung bei der genauere Informationen über das zu testende System vorliegen.
\subsection{OWASP Top 10}
	\large OWASP (Open Web Application Security Project)
	\large Goals:
	\begin{itemize}
		\item "identifying some of the most critical risks"
		\item "to raise awareness about application security"
		\item "not an application security program"
	\end{itemize}
	\large Top 10:
	\begin{enumerate}
		\item Injection
		\item XSS
		\item Broken Authentication and Session Management
		\item Insecure ddirect object references
		\item XS request forgery
		\item Security misconfiguration
		\item Insecure cryptographic storage
		\item Failure to restrict URL access
		\item Insufficient transport layer protection
		\item Invalidated redirects and forwards
	\end{enumerate}
\subsection{SQL Injection}
One of the most exploited vuln., despite being well known for almost a decade
\begin{itemize}
\item Problem: App doesn't properly separate database statements form user data
\item Impact: Read/Write access to database (sometimes even databases form other apps which are on the same server)
\item Tool support: sqlmap (can find injection, dump content, supports different DBMS)
\item Solution: Never trust user-supplied data - Separate query logic form user data
\item Good Practice: Prepared statements
\end{itemize}
\subsection{REST}
Representational State Transfer ist ein Programmierprinzip für Webanwendungen bei dem eine dynamsich generierte Seite den Inhalt einer serverseitigen Aktion darstellt. (Anzeigen von Suchergebnissen)
\large Eigenschaften die ein REST erfüllen muss:
\begin{itemize}
	\item Adressierbarkeit
	\item Underschiedliche Repräsentationen
	\item Zustandslosigkeit
	\item Operationen
	\item Verwenden von Hypermedia
\end{itemize}

\section{WebSec 2}
\subsection{SQL Injection}
$ //TODO sollte aehnlich zum punkt sql inj weiter unten sein $
\subsection{XSS}
$ //TODO $

\section{Security Challanges}
\subsection{Attackermodel}
$ //TODO $
\subsection{Standartmodel}
$ //TODO $
\subsection{Dolyveyao}
$ //TODO $
\subsection{NithemSchroeder PublicKey}
$ //TODO mit SYNATX $
\subsection{Attacks Generel}
$ //TODO $
zb Social Eng ...
\subsection{MathShit ...}
$ //TODO fuer jedes X git es ein ... $
\subsection{Muenzwurf}
$ //TODO $
\subsection{Dataflow}
$ //TODO $
\subsection{Polices}
$ //TODO $
\subsection{Examples for Non-Modularity}
$ //TODO $

\section{SmartGrid}
\subsection{Szenario}
$ //TODO $
\subsection{Which Attacks are Posible}
$ //TODO Only highlevel$
\subsection{Abrechnung und Steuerung im Privathaus}
$ //TODO $

\section{Internet of Things}
\subsection{Applikationbeispiele}
$ //TODO $
\subsection{Sec Challanges}
$ //TODO combine the following points better // nicht gar so wichtig sind aber par sachen doppelt vom inhalt her$
\begin{itemize}
\item larger attack surface
\begin{itemize}
\item large number of interfaces
\item new possibilities of exploiting existing vuln. and threats
\item new interaction possibilities
\item new possible attack patterns or procedures, threats and damages
\item IoT devices use local and remote services and provide services with or without any human intervertion
\end{itemize}
\item Fault Tolerance
\begin{itemize}
\item Connectivity is not constant
\item context of devices changes over timem sometimes abruptly
\item system must cope with these changes, providing a partial service until the conditions are more facorable
\end{itemize}
\item Efficient Cryptographic Primitces
\begin{itemize}
\item provide good secirity primitives suitable for low resource consumption(energy, time, sace)
\item deployment of efficient key distribution and management systems
\end{itemize}
\item Integrate IoT with the rest of the Internet
\begin{itemize}
\item use protocols and security mechanisms used in the internet or bridge to them, archieving end-to-end security when translating the different underlying protocols, existence of such sub-networks and interactions between them must be taken into account
\end{itemize}
\item Intrusion Detection Systems and Survivability Mechanisms
\begin{itemize}
\item react to changes in environment
\end{itemize}
\item Trust Management and Secure Collaboration
\begin{itemize}
\item Need trust relationships between users and devices
\item Devices might know each other, or might be complete stranges
\item Need to interoperate the security policies of different components(how will devices react if a particular event arises?, Who are the elements i should collaborate with, and how?)
\end{itemize}
\item Models and mechanisms for device and data ownership
\begin{itemize}
\item Users may utilize their devices to authenticate themselves
\item Devices might perform certain operations on behalf of their users
\item MEchanisms and security policies and mechanisms that control how the data is created, accessed and protected
\end{itemize}
\item Software maintenance process
\begin{itemize}
\item How to release security patches
\end{itemize}
\item Usability
\begin{itemize}
\item security for IoT-based applications must be understandable and manageable by end-users
\end{itemize}
\item Privacy
\begin{itemize}
\item "Things" belong to people and collect information about actions of them
\item Devices and interfaces should do not leak personal information aboit the users location activities preferences // haha nice try NSA :D
\item Need to avoid that a communication partner is able to collect large amounts of information. Current Algorithms do not prevent this type of attack
\end{itemize}
\end{itemize}

\subsection{Sec Solutions}
\begin{itemize}
\item Fast and secure network bootstrapping
\begin{itemize}
\item minimizing security attacks when network is initialized
\item objects don't have knowledge regarding environment and their neighbouring nodes
\end{itemize}
\item Secure and Context-aware dynamic auto configuration
\begin{itemize}
\item When a new Object enters its security settings and its config will be automatically distributed via the network
\end{itemize}
\item Self-management and self-monitoring mechanisms
\begin{itemize}
\item ensuring smooth operations and continuous monitoring to track problems
\item trigger self-healing algorithms for fault avoidance
\end{itemize}
\item Creation of a Run-time Security Reconfigurability Platform
\begin{itemize}
\item allowing the end-users app or service to request particular security servies or settings
\end{itemize}
\end{itemize}

\subsection{RFID}
\begin{itemize}
\item Passive Tags - no power source
\item small amount of data
\item RFID reader powers tag, extracts data via radion
\end{itemize}
\large Applications
\begin{itemize}
\item Authentication Authorization Access Control
\begin{itemize}
\item Patient and practitioner Identification Hospital
\item Car Keys
\item Passports
\item Secure Entry cards
\item Libary books
\end{itemize}
\item Supply chain managment (inventory control)
\item Payment systems
\begin{itemize}
\item I-Pass
\item Credit Card
\end{itemize}
\item Animal Tracking - and Human 
\begin{itemize}
\item VeriChip: Human implantable RFID tag, can penetrate mud, blood ,water, only the size of a rice grain
\item Emergency access to patient-supplied health information
\item Portable medical records access
\item Disease/treatment management of at-risk population
\end{itemize}
\end{itemize}
\large Conderns:
\begin{itemize}
\item Problem: "unintended consequences"
\item Read tags through briefcases, etc
\item Exp: Tracking books. Can we figure out what you're reading, and where you are?
\end{itemize}
\large Privacy Threats
\begin{itemize}
\item Tracking: Determine where individuals are or where they have been
\item Hotlisting: Single out certain individuals because of the items they possess
\item Profiling: Identifying the items an idividual has in their possession
\end{itemize}
A sufficiently powerful directerd reader reads tags in a house or car, enabling large scale tracking and profiling of individuals \\ \newline
\newbox
\large Design Principles for Privacy
\begin{itemize}
\item Proportionality
\begin{itemize}
\item a balanced analysis of whether the risk to individuals is sufficiently mitigated to justify the use of RFID
\end{itemize}
\item Transparency
\begin{itemize}
\item ensure RFID is not secretly used to collect data (limited to a very close read range and crypto protection)
\end{itemize}
\item Notice
\begin{itemize}
\item individuals in RFID enabled environments should receive notification that the tech. is used, type of data collection, and how the data will be shared and used
\end{itemize}
\item Choice
\begin{itemize}
\item User has choice to disable tag at some point(if the tag is in an item the individual will carry with them)
\item disabling with zb. a blocker bag
\end{itemize}
\end{itemize}


\subsection{Baumbasierendes Model}
fuer RFID
$ //TODO $
\section{6 Threads Landscape}
\subsection{Phishing}
$ //TODO $
\subsection{Spearphising}
$ //TODO $
\subsection{Zertificate}
$ //TODO $
\subsection{Chain of Trust}
$ //TODO $

\section{Cause Analysis}
\subsection{Implementation Errors}
$ //TODO $
die ueber legitime Ports ausgenutzt werden koennen.

\section{Product ...}
\subsection{Common criteria}
$ //TODO $
\subsection{Was ist ein Sec Target}
$ //TODO $
\subsection{Was ist ein Connection Profile}
$ //TODO $
\subsection{Security function requ}
$ //TODO $
\subsection{Security assurance requ}
$ //TODO $
\subsection{Was sind EALs}
$ //TODO $

\section{WebVulnerabilities}
\subsection{State of the art}
$ //TODO $
\subsection{Top 10}
$ //TODO sollte sehr aehnlich zu owasp top 10 weiter oben sein$
\subsection{SQL Injection Counter MEasures}
SYNTAX
Some Stuff in section websec 
$ //TODO add syntax $
$ //TODO add example Counter Measures $
\subsection{XSS - Cross-site scripting}
\begin{itemize}
\item User injects JavaScript code that subsequentially gets executed in the browser of the visitor
\item Problem: User data that contains HTML markup is not properly escaped, the browser renders the attacker code as port of the web page
\end{itemize}

\subsection{CSRF}
\subsection{Session Hacking}
//Broken Authentication and session mgmt.
\begin{itemize}
\item Badly implemented access control: index.html redirects to secret.html if the password is correct
\item Sometimes one can go directly to secret.html without password check
\end{itemize}

\section{Formal Security Models}
\subsection{Welche art von Modellen gibt es}
$ //TODO $
\subsection{Ansatz und Vorteile von Formal Security}
$ //TODO $
\subsection{Conclusions}
$ //TODO $

\section{BufferOverflow}
\subsection{Buffer Overflows}
Things good to know:
\begin{itemize}
\item First publication 1972
\item First documented exploit, Morris Worm 1988
\item Three famous Worms(2001 Code Red Worm, 2003 SQL Slammer, 2008 Conficker Worm)
\item Triggered by external data input with Dangerous code
\item Mostly seen in C/C++ because of missing memory boundarys or access checking
\end{itemize}
\subsubsection{Stack Buffer Overflows}
Programm Stack: used for managing prog execution and prog state, saves 
\begin{itemize}
\item Return Address
\item Function Arguments
\item Local Varibales
\end{itemize}
on the Stack Frame.
Register EBP saves actual address of the Frame (Intel)
Frame Pointer is the Reference Pointer within the Stack
Stack gets modified during following actions
\begin{itemize}
\item Function call
\item Function Initialization
\item When function returns
\end{itemize}
$ //TODO merge FunctionCall example $
\subsubsection{Heap Buffer Overflows}
Ich denke hierzu haben wir nicht wirklich was gemacht weis nicht ob das jedoch bekannt ist

\subsection{Basic Errors}
\begin{itemize}
\item Strings
\begin{itemize}
\item Unbounded String Copies
\item Off-by-One Errors
\item Null-Termination Errors
\item String Truncation
\end{itemize}
\item Memory Managment Errors
\begin{itemize}
\item Double Free
\end{itemize}
\item Integer
\begin{itemize}
\item Integer Overflow
\item Sogn Errors
\item Truncation Errors
\end{itemize}
\item Formated Output
\begin{itemize}
\item Format String
\end{itemize}
\end{itemize} 

\subsection{Code Injection}
\begin{itemize}
\item Attacker creats malicious argument
\item The argument is a string which includes a pointer to the malicious code
\item When the function return, the maliciouse code is executed
\item The malicious code has the same permissions as the program exploited
\end{itemize}
\large Malicious Argument
\begin{itemize}
\item Program must interpret the input as legitimate input
\item The argument must allow the attacker to execute his code
\item The argument should not crash the attacked program
\end{itemize}
\large Malicious Code
\begin{itemize}
\item Code can be part of the argument but mostly is injected through a different vector
\item Code has the ability to execute every functionality which can be programmed; mostly remote shell on targeted system
\end{itemize}

\subsection{Return to libc}
\begin{itemize}
\item In the Return to libc the attacker modifies the Return Address to point to a function which is already in the address space(libc)
\item This allows to inject new functionality to be executed during the normal application execution.
\item Injecting addresses of existing functions e.g. system() exec()
\end{itemize}
Exploit
\begin{enumerate}
\item Overwrite the return addresse with an address of a existing libc function
\item Create a Stack Frame to execute other functions
\item Restore the original Stack Frame to execute the original application
\end{enumerate}
Interesting, because:
\begin{itemize}
\item Attacker executes function in a chain
\item Attack is difficult to detect (application is not crashing)
\item No code is infected
\item Easy circumvention of memory protection mechanisms
\end{itemize}

\subsection{Mitigation}
\begin{itemize}
\item NX-Bit
\begin{itemize}
\item \textbf{N}o e\textbf{X}ecution Bit 
\item Hardware- and Software-based method; Operating System support necessary
\item Strict separation between code and data
\item Code which is written to data segments during Buffer Overflow are not executed
\item Does not prevent the Buffer Overflow
\item Can be circumvented by Return-to-libc
\end{itemize}
\item Address Space Layout Randomization
\begin{itemize}
\item Executed program is put into random address range(Stack, Heap, Libaries)
\item Prevents attacks such as "return-to-libc" or "code injection"
\item Operating System must support this: (Linux since Kernel 2.6.12, Windows Vista, Mac OS X)
\item ASLR can be targeted using "Spraying"
\end{itemize}
\item Canaries
\begin{itemize}
\item Canaries are known valuesm which are placed between the Stack Frame and the return address
\item During Buffer Overflow the Canary is modified before the return address 
\item A not known value in the Canary memory segment means an external modification
\item \textbf{Result:} Program is terminated
\item Types of Canarys
\begin{enumerate}
\item Terminator Canraries
\item Random Canraries
\item Random XOR Canraries
\end{enumerate}
\item Does not prevent Heap Buffer Overflows
\item Supported by: GCC , Microsoft Visual Studio
\item Usage of Canaries may result in performance issues
\end{itemize}
\item Secure Functions
\end{itemize}

\section{NOT RELEVANT}
\begin{tabular}{|c|c|} \hline
\textbf{Section} & \textbf{Thema} \\ \hline
Internet of Things & Geo Privacy \\ \hline
5 & Best Practice \\ \hline
5 & Risk for Tests \\ \hline
WebVul & extra Slides am ende \\ \hline
Smart Metering & * \\ \hline
Formal Sec Mod & Siemens zeug \\ \hline



\end{tabular}
\end{document}
